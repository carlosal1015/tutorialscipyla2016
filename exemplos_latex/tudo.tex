\documentclass{article}
\usepackage[utf8]{inputenc}
\usepackage[portuguese]{babel}

\author{Melissa Weber Mendonça}
\title{Exemplo de numeração automática\\e outras vantagens do \LaTeX}

\begin{document}
\maketitle

%\chapter{Introdução}

Aqui, vamos ver exemplos de numeração automática no \LaTeX. 

\section{Primeira seção\label{primeira}}

Esta é a primeira seção. Mais informações na Tabela~\ref{tab:distros}.

\begin{table}
   \begin{center}
      \begin{tabular}{|c|c|c|}
        \hline
        Distros & \textbf{Gnome} & \textbf{KDE} \\\hline
        Ubuntu & X & \\\hline
        Fedora & X & \\\hline
        Debian & X & \\\hline
        OpenSUSE & & X \\\hline
        Slackware & & X\\\hline
      \end{tabular}
   \end{center}
   \caption{\label{tab:distros}Tabela sobre distribuições Linux e seus ambientes gráficos.}
\end{table}

%\section{Mais uma!!!}

% \section{Outra seção!}

% Esse texto é novo. Vem da Seção \ref{primeira}.

\section{Segunda seção}

Esta é a segunda seção.

\subsection{Subseção}

Surpresa! Uma subseção.

\end{document}
